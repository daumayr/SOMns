\documentclass[a4paper,UKenglish,cleveref,autoref]{lipics-v2019}
%This is a template for producing LIPIcs articles.
%See lipics-manual.pdf for further information.
%for A4 paper format use option "a4paper", for US-letter use option "letterpaper"
%for british hyphenation rules use option "UKenglish", for american hyphenation rules use option "USenglish"
%for section-numbered lemmas etc., use "numberwithinsect"
%for enabling cleveref support, use "cleveref"
%for enabling cleveref support, use "autoref"


\include{scripts/init}
\include{gen/performance}
\usepackage{graphicx}

\newcommand{\citep}[1]{\,\cite{#1}}
\newcommand{\citet}[1]{\,<Name>\,\cite{#1}}

\def\SOMns{SOM{\sc ns}\xspace}
\def\AcmeAir{Acme\,Air\xspace}
\def\Kompos{Kómpos\xspace}

%\graphicspath{{./graphics/}}%helpful if your graphic files are in another directory

%\bibliographystyle{plainurl}% the mandatory bibstyle

\title{Uniform Record \& Replay for Multi-paradigm Concurrent Programs }

%\titlerunning{Dummy short title}%optional, please use if title is longer than one line


\author{Anonymous}{Anonymous}{}{[orcid]}{[funding]}
%% commented for submission:
%\author{Dominik Aumayr}{Johannes Kepler University, Linz, Austria}{dominik.aumayr@jku.at}{https://orcid.org/0000-0002-1825-0097}{}
%\author{Stefan Marr}{University of Kent, Canterbury, United Kingdom}{s.marr@kent.ac.uk}{https://orcid.org/0000-0002-1825-0097}{}
%\author{Elisa Gonzalez Boix}{Vrije Universiteit Brussel, Brussel, Belgium}{egonzale@vub.be}{https://orcid.org/0000-0002-1825-0097}{}
%\author{Hanspeter M\"{o}ssenb\"{o}ck}{Johannes Kepler University, Linz, Austria}{hanspeter.moessenboeck@jku.at}{https://orcid.org/0000-0002-1825-0097}{}

\authorrunning{Anonymous et al.}
%% commented for submission:
%\authorrunning{D. Aumayr, S. Marr, E. Gonzalez Boix and H. M\"{o}ssenb\"{o}ck}%TODO mandatory. First: Use abbreviated first/middle names. Second (only in severe cases): Use first author plus 'et al.'

\Copyright{Anonymous}
%% commented for submission:
%\Copyright{Dominik Aumayr, Stefan Marr, Elisa Gonzalez Boix and Hanspeter M\"{o}ssenb\"{o}ck}%TODO mandatory, please use full first names. LIPIcs license is "CC-BY";  http://creativecommons.org/licenses/by/3.0/

%ACM classification no needed for submission but I left it otherwise it complains
\ccsdesc[300]{Software and its engineering~Concurrent programming languages}
\ccsdesc[300]{Software and its engineering~Software testing and debugging}

\keywords{Debugging, Tooling, Concurrency, Record \& Replay, Multi-Paradigm}

\category{}%optional, e.g. invited paper

\relatedversion{}%optional, e.g. full version hosted on arXiv, HAL, or other respository/website
%\relatedversion{A full version of the paper is available at \url{...}.}

\supplement{}%optional, e.g. related research data, source code, ... hosted on a repository like zenodo, figshare, GitHub, ...

%\funding{(Optional) general funding statement \dots}%optional, to capture a funding statement, which applies to all authors. Please enter author specific funding statements as fifth argument of the \author macro.

%% commented for submission:
%\acknowledgements{This research is funded in part by a collaboration grant of the Austrian Science Fund (FWF) and the Research Foundation Flanders (FWO Belgium) as project I2491-N31 and G004816N.}%optional

%\nolinenumbers %uncomment to disable line numbering

%\hideLIPIcs  %uncomment to remove references to LIPIcs series (logo, DOI, ...), e.g. when preparing a pre-final version to be uploaded to arXiv or another public repository

%Editor-only macros:: begin (do not touch as author)%%%%%%%%%%%%%%%%%%%%%%%%%%%%%%%%%%
\EventEditors{Robert Hirschfeld}
\EventNoEds{1}
\EventLongTitle{34th European Conference on Object-Oriented Programming (ECOOP 2020)}
\EventShortTitle{ECOOP 2020}
\EventAcronym{ECOOP}
\EventYear{2020}
\EventDate{July 13--17, 2020}
\EventLocation{Berlin, Germany}
\EventLogo{}
\SeriesVolume{42}
\ArticleNo{1}
%%%%%%%%%%%%%%%%%%%%%%%%%%%%%%%%%%%%%%%%%%%%%%%%%%%%%%

\begin{document}

\maketitle


\begin{abstract}

\end{abstract}



\section{Evaluation of Uniform Record \& Replay}
This section evaluates the flexibility and generality of our approach.
% While our approach is designed with uniformity in mind,  flexibility was also an important design goal.
It should be possible to support
a wide range of concurrency models,
vary the strategy of how to record \& replay a concurrency model,
and the way how to model events
without having to change the trace format or corresponding infrastructure.

\section{Evaluation of Performance}

This section evaluates the recording performance
of our uniform record \& replay implementation.
The overall goal of this evaluation is to assess whether our approach
is practical, and how it compares to a special-purpose implementation.

To assess whether our approach is practical, we measure the run-time overhead
of recording for each model individually.
The fork/join model is not evaluated
as it does not introduce events
that need to be recorded and replayed.
In particular, we assess whether using our uniform record \& replay approach
introduces an overhead too high for use for diagnostic purposes in production
scenarios.
Depending on the specific application, an overhead of 10\% may still
be considered acceptable,
while doubling the run time may disqualify the approach from consideration.

To assess how our approach compares to a special-purpose implementation, 
we use the work of 
Aumayr\,et\,al which developed a record \& replay system for \SOMns.
That work reports optimizations for their system
with a minimal run-time overhead for a real-world application.
As such, we assume it to be a highly optimized system
and a good approach to compare to since it also builds on \SOMns.
%At the same time, we assume that a uniform approach needs to reach
%similar performance as a special-purpose implementation
%to be considered acceptable from a performance perspective.

\subsection{Methodology}

% EGB: I moved the following text up to the glue text.
%As mentioned, the main focus of this evaluation is to compare
%the performance overhead of our approach
% of our uniform record \& replay system for the CEL model
%with a special-purpose implementation. 
%For CEL, we use the work of 
%Aumayr\,et\,al.\citep{Aumayr:2018:Replay} since it reports optimizations for their system
%with a minimal run-time overhead for a real-world application.
%As such, we assume it to be a highly optimized system
%and a good comparison since it also builds on \SOMns.

We now explain the methodology to select benchmarks as well as details of the software and hardware used for their execution.

\paragraph*{Benchmark Selection}

We evaluate the recording performance of our implementation in \SOMns based on different benchmark suites for each model.
%For the evaluation of
%recording performance
For the CEL model,
we rely on the widely used Savina benchmark suite
for actors.
% SM: I think my edit before sets this up more directly, and less as a consequence.
% As the Savina benchmarks were used
% by a specialized record \& replay
% for actors in the same VM,
% this allows for a direct
% comparison of recording performance.
%
For threads \& locks and STM
we use the LeeTM benchmark,
the Vacation benchmark from the STAMP suite,
and a variant of the classic dining philosophers.
%
Lee and Vacation were designed for
transactional systems.
Thus, for our threads \& locks version,
we replaced transactions
with mutual exclusion through locks.
%
The benchmarks used for CSP
are based on the
corresponding Savina benchmarks.


For the CEL model, \SOMns comes with a comprehensive set of benchmarks
enabling us to do a detailed evaluation.
However, for the other concurrency models,
% \SOMns does not come with a similarly comprehensive set of benchmarks.
we ported them from various sources.
Furthermore, \SOMns' STM implementation is not as optimized as the other concurrency models.
This means a performance evaluation is merely an indication
that our approach works.
It will not indicate how the performance overhead for an optimized system would look like.

\paragraph*{Benchmark Execution}

We argue that \SOMns is a suitable platform for such evaluation of
run-time overhead as it has performance
comparable to NodeJS
which is used in production systems.
It achieves this performance by using
the Graal just-in-time compiler on top of the GraalVM.
%
Because of this run-time profiling and compilation, benchmarks
do need time to \emph{warm up},
which depending on the benchmark, may take
multiple iterations before performance stabilizes.
%
We account for this warmup behavior
by executing each benchmark
for 1000 iterations.
%
After manual inspection,
we discarded the first 250
iterations as warmup.
%
The remaining 750 iterations
of each benchmark
are assumed to be representative
for the performance behavior of a longer running application.

For all benchmarks, we report the \emph{run-time factor} scaled based on the 
performance of the baseline without recording support
as it allows us to compare the results intuitively.
The run-time factor is a simple scaling transformation, which does not effect
statistic properties of the data, similar to a transformation from \textdegree{}C to \textdegree{}F.
It scales each measurement to the mean measurement for the benchmark on \SOMns
without recording support.

The benchmarks were executed on
a machine with a quadcore
Intel Core i7-6700 CPU, 3.40GHz, with 16GB RAM, 
a 256GB SSD, Ubuntu 18.04.3 LTS (kernel 4.15.0-72),
and a custom built OpenJDK 1.8.0\_232
with JVMCI and Graal version 19.3-b04.


In multithreaded execution,
the performance of a benchmark
may be sensitive to contention.
%
For instance in the Philosophers benchmark,
activities compete for access to forks,
depending on the order in which
philosophers acquire and release forks
the number of failed attempts
and the time the benchmark takes may increase.
%
As our tracing of nondeterministic events
inevitably is going
to have an effect on
the execution,
we restrict execution
to one thread
to obtain more stable results
that reflect the actual tracing overhead.
%
The single threaded
execution ensures
that the tracing overhead
is not hidden, e.g., by thread contention.
%
However, for the CSP benchmarks,
we had to use at least 2 threads
for execution, since the processes are directly mapped to threads
and channels have rendezvous semantics.
%
The CSP benchmarks do not have contention
on the channels when only 2 threads are used.
Therefore, any overhead of tracing should still be measurable.

\subsection{Recording Performance for CEL}
\label{sec:evalActors}

The evaluation of the recording overhead on the CEL model is based on the Savina actor benchmark suite.
\cref{fig:actor-perf} shows the recording overhead for the benchmarks, \ie the run-time factor for all benchmarks.
%
The geometric mean of the run-time factors (red dots in the plot)
indicates an overall overhead of\SavinaAvgOverheadP
for recording nondeterministic events for the CEL model,
where a majority (16 of 20) of the Savina benchmarks
have an overhead
in the 0\,to\,10\% range.
%
ForkJoinThroughput has the highest overhead with\SavinaMaxOverheadP.
The Trapezoidal benchmark has the lowest overhead
of\SavinaMinOverheadP
compared to \SOMns without recording support.
%
Trapezoidal's `speedup' is insignificant as can be seen by the overlapping boxplots.

Our results indicate that
the recording overhead
depends on the mix of operations performed by an application.
Benchmarks with a high proportion of nondeterministic events
such as ForkJoinThroughput,
which just creates actors and sends messages to them,
or AStarSearch and LogisticsMapSeries,
which have large number of actors communicating,
have higher overhead than benchmarks with few nondeterministic events
such as Trapezoidal, NQueens, and even ThreadRing.
For instance, Trapezoidal is one of the more computationally intensive
benchmarks of the Savina suite,
and thus, has a comparably small number of nondeterministic events that are recorded.


\begin{figure*}
	\begin{minipage}{.475\textwidth}
		%\centering
    % \PerformanceActors
    \includegraphics{gen/savina-performance-actors-1.pdf}
		\caption{
      Recording performance of uniform record \& replay for CEL actors
      compared to regular benchmark executions.
      The average overhead is\SavinaAvgOverheadP (min.\SavinaMinOverheadP, max.\SavinaMaxOverheadP) and thus,
      in an acceptable range for many production applications.
      \vspace{2\baselineskip}}
		\label{fig:actor-perf}
	\end{minipage}
	\hspace{0.05\textwidth}
	\begin{minipage}{.475\textwidth}
		%\centering
    % \PerformanceActorsOld
    \includegraphics{gen/savina-performance-old-1.png}
		\caption{
      Comparison of the recording performance of our uniform record \& replay
      with a solution optimized for actors only,
      both to the baseline of \SOMns without any recording.
      The actors-specific record \& replay
      has an average overhead of\SavinaOldAvgOverheadP
      (min.\SavinaOldMinOverheadP, max.\SavinaOldMaxOverheadP).
      This means our uniform solution is
      competitive with the optimized actors-specific solutions.}
		\label{fig:actor-perf-comp}
	\end{minipage}
\end{figure*}


\subsection{Comparison with Specialized CEL Record \& Replay}

We now compare our uniform record \& replay for CEL actors with a specialized CEL record \& replay system for \SOMns\citep{Aumayr:2018:Replay}.
% developed 
%a record \& replay system for \SOMns that
%is dedicated to the CEL model (cf. \cref{sec:RW:ActorRP}).
%
%As they reported on optimizations,
%the focus of their work was primarily performance.
%
While we cannot assume that their performance is perfect,
they reported optimizations to minimize the run-time overhead
of record \& replay for actors.
%
Since both approaches are implemented
for \SOMns, we can use the Savina actor benchmarks and directly
compare the performance of our uniform approach
with the performance
of this special-purpose solution.

\paragraph*{Run-time Performance}
\Cref{fig:actor-perf-comp} shows a comparison of the recording performance for both approaches,
normalized to untraced executions of Savina benchmarks,
%
Due to the characteristics of the benchmarks,
the approaches perform differently on each benchmark.
%
Overall, the recording overhead of both approaches
is in a similar range.
%
The actors-specific record \& replay 
has an average overhead of\SavinaOldAvgOverheadP (min.\SavinaOldMinOverheadP, max.\SavinaOldMaxOverheadP),
while our unified approach
had an overhead of\SavinaAvgOverheadP (min.\SavinaMinOverheadP, max.\SavinaMaxOverheadP).
This indicates that our uniform approach
is competitive with an optimized special-purpose implementation.

\paragraph*{Trace Size}
\label{sec:eval-trace-size}

In \cref{tab:tracesize}, the size of the traces produced by both approaches
are compared. 
The trace sizes of our uniform approach
fall within a similar range as the trace sizes of the actor-specific approach.
Directly compared to each other, the uniform approach has an average trace size overhead of\TraceAvgOverheadP (min.\TraceMinOverheadP, max.\TraceMaxOverheadP).
%
The high variation of the trace size factor indicates
that benchmark characteristics
influence how compact the nondeterminism can be captured.
%
We conclude that the uniform record \& replay
can reach performance and trace sizes competitive
with approaches that aim at a single model.
Thus, our approach can be as compact as special-purpose approaches.
% \egb{Stefan, can you add here emphasis on a compact trace?}

%\sm{We should emphasize here that `the others' report on optimizations
%and thus, their work was primarily focused on performance.
%While we cannot assume that their performance is perfect,
%we assume that they attempted to minimize run-time overhead.}


\begin{figure*}
	\begin{minipage}{.475\textwidth}
		%\centering
		\captionsetup{type=table}
		\caption{
      Size of the traces recorded per benchmark iteration.
      The Factor is the trace size of the Uniform approach
      normalized to the Specialized approach.
      On average, our uniform approach has a trace size overhead
      of\TraceAvgOverheadP (min.\TraceMinOverheadP, max.\TraceMaxOverheadP),
      and is thus competitive with a special purpose approach.
    }
		\resizebox{\textwidth}{!}{%
			\SavinaTraceDataTable}

		\label{tab:tracesize}
	\end{minipage}
	\hspace{0.05\textwidth}
	\begin{minipage}{.475\textwidth}
		%\centering
		\PerformanceCSP
		\vspace{-6mm}
		\caption{Recording performance of uniform record \& replay
      for a subset of the Savina benchmarks modified for CSP.
      The average overhead is\CSPAvgOverheadP
      (min.\CSPMinOverheadP, max.\CSPMaxOverheadP).}
		\label{fig:csp-perf}
		\vspace{5mm}
		\PerformanceMutex
		\vspace{-6mm}
		\caption{Recording performance of our prototype for 
      Threads \& Locks compared to untraced execution.
      The average overhead is\MutexAvgOverheadP
      (min.\MutexMinOverheadP, max.\MutexMaxOverheadP).}
		\label{fig:mutex-perf}
		\vspace{5mm}
		\PerformanceSTM
		\vspace{-6mm}
		\caption{The STM implementation used in \SOMns is not optimized.
       The tracing performance is not representative of optimized
       implementations.}
		\label{fig:stm-perf}
	\end{minipage}
\end{figure*}

\subsection{Recording Performance for Threads \& Locks}

The evaluation for recording overhead on the T\&L model is based on benchmarks
designed to compare locking with STM, namely the LeeTM and Vacation benchmarks,
both about 600--700 LOC each.
We also ported the Philosophers benchmark.
The benchmarks were intended to measure how the nondeterministic program parts
are synchronized, and how the synchronization overhead changes performance.
In our evaluation, the synchronization overhead is not relevant,
however, the nondeterministic events originate from the same program elements,
and thus, allow us to measure their overhead.

\Cref{fig:mutex-perf} shows the recording performance
for our benchmarks of threads \& locks.
The geometric mean of the run-time factors
indicates an overall overhead of\MutexAvgOverheadP
(min.\MutexMinOverheadP, max.\MutexMaxOverheadP).
%
%
This means, the results are in a similar
range of overhead as the record \& replay
for the CEL model.

%% SM: removed, I don't know how to integrate this. It doesn't fit here, and it doesn't fit in the STM section.
% For the Vacation benchmark, performance differences to the
% STM version can in part be explained
% with the different granularity of synchronization,
% as instead of few transactions
% many lock acquisitions
% are used to achieve a similar result.

\subsection{Recording Performance for CSP}

The evaluation of the recording overhead on the CSP model is based on benchmarks adapted from
the Savina benchmark suite.
While programs designed for CSP may take different shapes,
the benchmarks show a variety of message passing behaviors
that can also appear in more idiomatic CSP programs.

The recording performance for our CSP benchmarks,
as shown in \cref{fig:csp-perf},
results in an average (geometric) overhead of\CSPAvgOverheadP
(min.\CSPMinOverheadP, max.\CSPMaxOverheadP).
%
The maximum overhead of\CSPMaxOverheadP occurs 
in the FJThrouhput benchmark (red dot),
this is consistent with our results
for the recording performance in the actor model,
where FJThroughput is the benchmark with the
highest overhead, too.

% \smdone{this is not supported by the plot.
% You report here a maximum overhead, possibly based on some mean.
% But the plot only shows the median, where PingPong has the highest overhead.
% Which is not the same as for the actor model.}
% \dau{I guess this is resolved with the added dots for mean}

\subsection{Recording Performance for STM}

The evaluation of the recording overhead on the STM model is based on the same benchmarks
as threads \& locks.
The benchmark results are shown in \cref{fig:stm-perf},
the geometric mean of the run-time factors
indicates an average overhead of\STMAvgOverheadP
(min.\STMMinOverheadP, max.\STMMaxOverheadP).
%
These numbers are unfortunately not generalizable.
The STM implementation of \SOMns is not optimized for performance.
While the other concurrency models execute highly optimized code,
the STM is currently about 5x slower than sequential code.
This means, the STM itself hides any overhead from the recording.

% %
% Interpretation of the results
% has to take into consideration
% that the STM implementation of \SOMns
% is not highly optimized,
% \ie in an optimized STM system
% the observed overheads are likely to be higher.
%
However, our approach to recording for the STM 
records only a single event per successfully committed
transaction.
Thus, if we had applied it to a highly optimized STM,
it would have a small constant overhead for each commit operation,
which would likely be in the low percentage range
for applications with large number of transactions.


%%
%% Bibliography
%%

%% Please use bibtex,

%\bibliography{paper}

% \appendix
% \section{Appendix}


\end{document}
